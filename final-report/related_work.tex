\section{Related Work}
\label{sec:related_work}

Previous approaches to determine the relative position and rotation or calibration between a camera and IMU have been labor intensive, required special equipment, or limited to calibration for the rotaiton only. These previous approaches include measuring by hand, using estimated values from a CAD program during mechincal design, or surveying techniques where the physical housing of the IMU and camera are measured by a laser scanner \cite{Johnson_fieldtesting}.  More interesting calibration strategies have estimated calibration using data directly from the IMU and camera themselves. Lobo and Dias provide a rotation-only calibration toolbox called InerVis Matlab Toolbox \cite{InerVis}, which requires aligning a checkboard target with gravity and comparing the IMU's measurement of gravity. In later work, Lobo and Dias develop methods for the calibration of translation using specialized equipment such as a turntable  \cite{lobo2007relative}. Unfortunately this approach requires the user to rotate the whole IMU and camera system around the IMU's center. Practically this can be very difficult for systems mounted on vehicles. More recent work has determined the relative rotation and translation as part of a filtering framework using only a standard checkboard pattern.  Mirzaei and Roumeliotis \cite{mirzaei2008kalman} estimate the relative calibration parameters as one of their state variables in an extended Kalman filter. They iterate the correction step because of the high nonlinearity of the observation model. Kelly \cite{2011:kelly:article} addresses the high nonlinearity by proposing the use of an unscented Kalman filter. 